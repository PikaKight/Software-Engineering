\documentclass[12pt]{article}
\usepackage{tikz}
\usepackage{amssymb}
\usepackage{amsmath}
\usepackage{breqn}

\usetikzlibrary{automata, positioning, arrows}

\title{SE 3310 Theoretical Foundations of Software Engineering Assignment 3}
\author{Marcus Tuen Muk}
\date{March 02 2023}


\begin{document}


    \section{Pushdown Automaton}
        
        \subsection{Give the state-transition diagram of a pushdown automaton recognizing the following language.}
            \[a^nb^{(n+m)}c^m : n,m > 0\]

            \begin{figure}[ht]
                \centering
                \begin{tikzpicture}
                    \node[state, accepting, initial] (q0) {q0};
                    \node[state, right = 3cm of q0] (q1) {q1};
                    \node[state, below = 3cm of q1] (q2) {q2};
                    \node[state, below = 3cm of q2] (q3) {q3};
                    \node[state, accepting, left = 3cm of q3] (q4) {q4};
                                         

                    \draw   (q0) edge[->, right] node[above]{$\epsilon$, $\epsilon$ $\rightarrow$ $\$$} (q1)
                            (q1) edge[->, loop right] node[right]{a, $\epsilon$ $\rightarrow$ a} (q1)
                            (q1) edge[->, below] node[right]{b, a $\vert$ $\epsilon$  $\rightarrow$ b} (q2)
                            (q2) edge[->, loop right] node[right]{b, a $\vert$ $\epsilon$ $\rightarrow$ b} (q2)
                            (q2) edge[->, below] node[right]{c, b $\vert$ $\epsilon$ $\rightarrow$ $\epsilon$} (q3)
                            (q3) edge[->, loop right] node[right]{c, b $\vert$ $\epsilon$ $\rightarrow$ $\epsilon$} (q3)
                            (q3) edge[->, left] node[below]{$\epsilon$, $\$$ $\rightarrow$ $\epsilon$} (q4);
                \end{tikzpicture} 
            \end{figure}
\end{document}